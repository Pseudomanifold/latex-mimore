\section{Adding content}

This source is in a separate file to demonstrate how to \verb|include|
things. It is good style to let a new section begin on a new page. But
you do not have to do this, of course. If you write longer text, make
judicious use of paragraph breaks by adding newlines. For example, this
is the last line of the paragraph.

And now a new paragraph begins. The short indent of the new paragraph
makes it easier for a reader to perceive breaks in the text, but it is
not as harsh to the eye as a completely blank line. Do \emph{not} fiddle
with the indent or with the spacing between paragraphs!

Along with the standard environments, this template offers
\verb|paralist| for lists within paragraphs.
%
Here's a quick example: The American constitution speaks, among others, of
%
\begin{inparaenum}[(i)]
  \item life
  \item liberty
  \item the pursuit of happiness.
\end{inparaenum}
%
When writing a report, you hopefully have all of these.

\subsection{Citations \& bibliography}

Use the \verb|\autocite| command to cite literature. Do \emph{not} use
citations in lieu of nouns. Hence, the following is generally frowned
upon:
%
\begin{quote}
  As~\autocite{Edelsbrunner10} shows, \dots
\end{quote}
%
Instead, use this:
%
\begin{quote}
  As previously shown~\autocite{Edelsbrunner10}, \dots
\end{quote}
%
Or better:
%
\begin{quote}
  As Edelsbrunner and Harer~\autocite{Edelsbrunner10} showed, \dots
\end{quote}
%
You may also use special citation commands for the author names, e.g.\
\verb|\citet| or \verb|\citep|, but this guide prefers typing the author
names yourself.
%
It is also possible to use \verb|\autocites| to cite multiple authors.
So we could also talk about previous work by Edelsbrunner et
al.~\autocites{Edelsbrunner10, Edelsbrunner02}. Citations will be sorted
automatically.

Particular care should be take in order to properly format citations
that you download from somewhere. Even Google Scholar is known to
produce incorrect references.
%
When in doubt, consult the
documentation\footnote{\url{https://en.wikibooks.org/wiki/LaTeX/Bibliography_Management}}.
The bibliography of this template also contains some examples of proper
bibliography usage.

\subsection{Text in other languages}

Since this template uses \verb|babel| to support different languages, you can easily add ``foreign'' text by wrapping it in \verb|otherlanguage|:
%
\begin{quote}
  \begin{otherlanguage*}{ngerman}
    Uns ist in alten Geschichten viel Herrliches erzählt worden:
    von ruhmvollen Helden und ihren schweren Kämpfen, von
    höchstem Glück, von tiefstem Schmerz und von dem Heldenkampf
    der tapferen Burgunden könnt Ihr jetzt eine herrliche
    Geschichte vernehmen.
  \end{otherlanguage*}
\end{quote}
%
This works for all languages that have been used as optional arguments
in the inclusion of the \verb|babel| package. At present, this only
includes English~(the default language) and French:
%
\begin{quote}
  \begin{otherlanguage*}{french}
    Le roi Charles, notre empereur, le Grand, sept ans tous pleins
    est resté dans l'Espagne: jusqu'à la mer il a conquis la terre
    hautaine. Plus un château qui devant lui résiste, plus une muraille
    à forcer, plus une cité, hormis Saragosse, qui est sur une
    montagne. Le roi Marsile la tient, qui n'aime pas Dieu. C'est
    Mahomet qu'il sert, Apollin qu'il prie. Il ne peut pas s'en garder:
    le malheur l'atteindra. 
  \end{otherlanguage*}
\end{quote}
%
You may add other languages as well but should make sure that the main
language of the document is the \emph{last} one that is specified, as it
controls how things like the table of contents are named.

\subsection{Other resources}

Other resources comprise an excellent guide on how to write a seminar
report\footnote{\url{http://gvv.mpi-inf.mpg.de/teaching/how_to_thesis/how_to_write_a_report_slussalek.pdf}},
as well as Donald Knuth's lectures on mathematical
writing\footnote{\url{http://jmlr.csail.mit.edu/reviewing-papers/knuth_mathematical_writing.pdf}},
although this last guide is more relevant for writing about, well,
mathematics. There are also some good starting points about
\emph{writing} papers and \emph{reading} them by Bob
Laramee~\autocites{Laramee10,Laramee11}.
